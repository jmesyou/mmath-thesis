% uWaterloo Thesis Template for LaTeX 
% Last Updated June 14, 2017 by Stephen Carr, IST Client Services
% FOR ASSISTANCE, please send mail to rt-IST-CSmathsci@ist.uwaterloo.ca

% Effective October 2006, the University of Waterloo 
% requires electronic thesis submission. See the uWaterloo thesis regulations at
% https://uwaterloo.ca/graduate-studies/thesis.

% DON'T FORGET TO ADD YOUR OWN NAME AND TITLE in the "hyperref" package
% configuration below. THIS INFORMATION GETS EMBEDDED IN THE PDF FINAL PDF DOCUMENT.
% You can view the information if you view Properties of the PDF document.

% Many faculties/departments also require one or more printed
% copies. This template attempts to satisfy both types of output. 
% It is based on the standard "book" document class which provides all necessary 
% sectioning structures and allows multi-part theses.

% DISCLAIMER
% To the best of our knowledge, this template satisfies the current uWaterloo requirements.
% However, it is your responsibility to assure that you have met all 
% requirements of the University and your particular department.
% Many thanks for the feedback from many graduates that assisted the development of this template.

% -----------------------------------------------------------------------

% By default, output is produced that is geared toward generating a PDF 
% version optimized for viewing on an electronic display, including 
% hyperlinks within the PDF.
 
% E.g. to process a thesis called "mythesis.tex" based on this template, run:

% pdflatex mythesis	-- first pass of the pdflatex processor
% bibtex mythesis	-- generates bibliography from .bib data file(s)
% makeindex         -- should be run only if an index is used 
% pdflatex mythesis	-- fixes numbering in cross-references, bibliographic references, glossaries, index, etc.
% pdflatex mythesis	-- fixes numbering in cross-references, bibliographic references, glossaries, index, etc.

% If you use the recommended LaTeX editor, Texmaker, you would open the mythesis.tex
% file, then click the PDFLaTeX button. Then run BibTeX (under the Tools menu).
% Then click the PDFLaTeX button two more times. If you have an index as well,
% you'll need to run MakeIndex from the Tools menu as well, before running pdflatex
% the last two times.

% N.B. The "pdftex" program allows graphics in the following formats to be
% included with the "\includegraphics" command: PNG, PDF, JPEG, TIFF
% Tip 1: Generate your figures and photos in the size you want them to appear
% in your thesis, rather than scaling them with \includegraphics options.
% Tip 2: Any drawings you do should be in scalable vector graphic formats:
% SVG, PNG, WMF, EPS and then converted to PNG or PDF, so they are scalable in
% the final PDF as well.
% Tip 3: Photographs should be cropped and compressed so as not to be too large.

% To create a PDF output that is optimized for double-sided printing: 
%
% 1) comment-out the \documentclass statement in the preamble below, and
% un-comment the second \documentclass line.
%
% 2) change the value assigned below to the boolean variable
% "PrintVersion" from "false" to "true".

% --------------------- Start of Document Preamble -----------------------

% Specify the document class, default style attributes, and page dimensions
% For hyperlinked PDF, suitable for viewing on a computer, use this:
\documentclass[letterpaper,12pt,titlepage,oneside,final]{book}
 
% For PDF, suitable for double-sided printing, change the PrintVersion variable below
% to "true" and use this \documentclass line instead of the one above:
%\documentclass[letterpaper,12pt,titlepage,openright,twoside,final]{book}

% Some LaTeX commands I define for my own nomenclature.
% If you have to, it's better to change nomenclature once here than in a 
% million places throughout your thesis!
\newcommand{\package}[1]{\textbf{#1}} % package names in bold text
\newcommand{\cmmd}[1]{\textbackslash\texttt{#1}} % command name in tt font 
\newcommand{\href}[1]{#1} % does nothing, but defines the command so the
    % print-optimized version will ignore \href tags (redefined by hyperref pkg).
%\newcommand{\texorpdfstring}[2]{#1} % does nothing, but defines the command
% Anything defined here may be redefined by packages added below...

% This package allows if-then-else control structures.
\usepackage{ifthen}
\newboolean{PrintVersion}
\setboolean{PrintVersion}{false} 
% CHANGE THIS VALUE TO "true" as necessary, to improve printed results for hard copies
% by overriding some options of the hyperref package below.

%\usepackage{nomencl} % For a nomenclature (optional; available from ctan.org)
\usepackage{amsmath,amssymb,amstext} % Lots of math symbols and environments
\usepackage[pdftex]{graphicx} % For including graphics N.B. pdftex graphics driver 

% Hyperlinks make it very easy to navigate an electronic document.
% In addition, this is where you should specify the thesis title
% and author as they appear in the properties of the PDF document.
% Use the "hyperref" package 
% N.B. HYPERREF MUST BE THE LAST PACKAGE LOADED; ADD ADDITIONAL PKGS ABOVE
\usepackage[pdftex,pagebackref=false]{hyperref} % with basic options
		% N.B. pagebackref=true provides links back from the References to the body text. This can cause trouble for printing.
\hypersetup{
    plainpages=false,       % needed if Roman numbers in frontpages
    unicode=false,          % non-Latin characters in Acrobat’s bookmarks
    pdftoolbar=true,        % show Acrobat’s toolbar?
    pdfmenubar=true,        % show Acrobat’s menu?
    pdffitwindow=false,     % window fit to page when opened
    pdfstartview={FitH},    % fits the width of the page to the window
    pdftitle={uWaterloo\ LaTeX\ Thesis\ Template},    % title: CHANGE THIS TEXT!
%    pdfauthor={Author},    % author: CHANGE THIS TEXT! and uncomment this line
%    pdfsubject={Subject},  % subject: CHANGE THIS TEXT! and uncomment this line
%    pdfkeywords={keyword1} {key2} {key3}, % list of keywords, and uncomment this line if desired
    pdfnewwindow=true,      % links in new window
    colorlinks=true,        % false: boxed links; true: colored links
    linkcolor=blue,         % color of internal links
    citecolor=green,        % color of links to bibliography
    filecolor=magenta,      % color of file links
    urlcolor=cyan           % color of external links
}
\ifthenelse{\boolean{PrintVersion}}{   % for improved print quality, change some hyperref options
\hypersetup{	% override some previously defined hyperref options
%    colorlinks,%
    citecolor=black,%
    filecolor=black,%
    linkcolor=black,%
    urlcolor=black}
}{} % end of ifthenelse (no else)

\usepackage[automake,toc,abbreviations]{glossaries-extra} % Exception to the rule of hyperref being the last add-on package

\usepackage{minted}

\setminted[scala]{
	linenos,
	style=borland,
	obeytabs=true,
	tabsize=4,
	fontsize=\scriptsize,
	frame=lines,
	framesep=4mm,
	numbersep=-10pt
}
\setminted[java]{
	linenos,
	style=borland,
	obeytabs=true,
	tabsize=4,
	fontsize=\scriptsize,
	frame=lines,
	framesep=4mm,
	numbersep=-10pt
}
\setmintedinline{
	fontsize=\normalsize,
	escapeinside=\{\}
}

\newcommand{\scalainline}[1]{\mintinline{scala}|#1|}
\newcommand{\javainline}[1]{\mintinline{java}|#1|}

\def\CC{{C\nolinebreak[4]\hspace{-.05em}\raisebox{.4ex}{\tiny\bf ++}}}

\usepackage{caption}
\usepackage{subcaption}
\usepackage{epigraph}
\usepackage[section]{placeins}
\usepackage{svg}
\usepackage{smartdiagram}
\usepackage{pgfplots}

% If glossaries-extra is not in your LaTeX distribution, get it from CTAN (http://ctan.org/pkg/glossaries-extra), 
% although it's supposed to be in both the TeX Live and MikTeX distributions. There are also documentation and 
% installation instructions there.

% Setting up the page margins...
% uWaterloo thesis requirements specify a minimum of 1 inch (72pt) margin at the
% top, bottom, and outside page edges and a 1.125 in. (81pt) gutter
% margin (on binding side). While this is not an issue for electronic
% viewing, a PDF may be printed, and so we have the same page layout for
% both printed and electronic versions, we leave the gutter margin in.
% Set margins to minimum permitted by uWaterloo thesis regulations:
\setlength{\marginparwidth}{0pt} % width of margin notes
% N.B. If margin notes are used, you must adjust \textwidth, \marginparwidth
% and \marginparsep so that the space left between the margin notes and page
% edge is less than 15 mm (0.6 in.)
\setlength{\marginparsep}{0pt} % width of space between body text and margin notes
\setlength{\evensidemargin}{0.125in} % Adds 1/8 in. to binding side of all 
% even-numbered pages when the "twoside" printing option is selected
\setlength{\oddsidemargin}{0.125in} % Adds 1/8 in. to the left of all pages
% when "oneside" printing is selected, and to the left of all odd-numbered
% pages when "twoside" printing is selected
\setlength{\textwidth}{6.375in} % assuming US letter paper (8.5 in. x 11 in.) and 
% side margins as above
\raggedbottom

% The following statement specifies the amount of space between
% paragraphs. Other reasonable specifications are \bigskipamount and \smallskipamount.
\setlength{\parskip}{\medskipamount}

% The following statement controls the line spacing.  The default
% spacing corresponds to good typographic conventions and only slight
% changes (e.g., perhaps "1.2"), if any, should be made.
\renewcommand{\baselinestretch}{1} % this is the default line space setting

% By default, each chapter will start on a recto (right-hand side)
% page.  We also force each section of the front pages to start on 
% a recto page by inserting \cleardoublepage commands.
% In many cases, this will require that the verso page be
% blank and, while it should be counted, a page number should not be
% printed.  The following statements ensure a page number is not
% printed on an otherwise blank verso page.
\let\origdoublepage\cleardoublepage
\newcommand{\clearemptydoublepage}{%
  \clearpage{\pagestyle{empty}\origdoublepage}}
\let\cleardoublepage\clearemptydoublepage

% Define Glossary terms (This is properly done here, in the preamble. Could be \input{} from a file...)
% Main glossary entries -- definitions of relevant terminology

% Nomenclature glossary entries -- New definitions, or unusual terminology

% List of Abbreviations (abbreviations type is built in to the glossaries-extra package)
\newabbreviation{ast}{AST}{Abstract Syntax Tree}
\newabbreviation{tasty}{TASTy}{Typed Abstract Syntax Tree}
\newabbreviation{ir}{IR}{Intermediate Representation}
\newabbreviation{vm}{VM}{Virtual Machine}
\newabbreviation{jvm}{JVM}{Java Virtual Machine}
\newabbreviation{clr}{CLR}{Common Language Runtime}
\newabbreviation{llvm}{LLVM}{Low Level Virtual Machine}
\newabbreviation{jit}{JIT}{Just-in-time}
\newabbreviation{dsl}{DSL}{Domain Specific Language}
\newabbreviation{ssa}{SSA}{Static Single Assignment}

% List of Symbols
\newglossary*{symbols}{List of Symbols}
\newglossaryentry{phi}
{
name={$\phi$},
sort={label},
type=symbols,
description={Phi node}
}

\newglossaryentry{pi}
{
	name={$\pi$},
	sort={label},
	type=symbols,
	description={Pi node}
}

 
\makeglossaries

%======================================================================
%   L O G I C A L    D O C U M E N T -- the content of your thesis
%======================================================================
\begin{document}

% For a large document, it is a good idea to divide your thesis
% into several files, each one containing one chapter.
% To illustrate this idea, the "front pages" (i.e., title page,
% declaration, borrowers' page, abstract, acknowledgements,
% dedication, table of contents, list of tables, list of figures,
% nomenclature) are contained within the file "uw-ethesis-frontpgs.tex" which is
% included into the document by the following statement.
%----------------------------------------------------------------------
% FRONT MATERIAL
%----------------------------------------------------------------------
% T I T L E   P A G E
% -------------------
% Last updated June 14, 2017, by Stephen Carr, IST-Client Services
% The title page is counted as page `i' but we need to suppress the
% page number. Also, we don't want any headers or footers.
\pagestyle{empty}
\pagenumbering{roman}

% Source-guided Ad-hoc Data Representations in a Managed Runtime
% The contents of the title page are specified in the "titlepage"
% environment.
\begin{titlepage}
        \begin{center}
        \vspace*{1.0cm}

        \Huge
        {\bf Specializing Scala with Truffle}

        \vspace*{1.0cm}

        \normalsize
        by \\

        \vspace*{1.0cm}

        \Large
        James You \\

        \vspace*{3.0cm}

        \normalsize
        A thesis \\
        presented to the University of Waterloo \\ 
        in fulfillment of the \\
        thesis requirement for the degree of \\
        Master of Mathematics \\
        in \\
        Computer Science \\

        \vspace*{2.0cm}

        Waterloo, Ontario, Canada, 2022 \\

        \vspace*{1.0cm}

        \copyright\ James You 2022 \\
        \end{center}
\end{titlepage}

% The rest of the front pages should contain no headers and be numbered using Roman numerals starting with `ii'
\pagestyle{plain}
\setcounter{page}{2}

\cleardoublepage % Ends the current page and causes all figures and tables that have so far appeared in the input to be printed.
% In a two-sided printing style, it also makes the next page a right-hand (odd-numbered) page, producing a blank page if necessary.

% D E C L A R A T I O N   P A G E
% -------------------------------
  % The following is a sample Delaration Page as provided by the GSO
  % December 13th, 2006.  It is designed for an electronic thesis.
  \noindent
I hereby declare that I am the sole author of this thesis. This is a true copy of the thesis, including any required final revisions, as accepted by my examiners.

  \bigskip
  
  \noindent
I understand that my thesis may be made electronically available to the public.

\cleardoublepage

% A B S T R A C T
% ---------------

\begin{center}\textbf{Abstract}\end{center}

Scala is a multi-paradigm programming language with higher-order abstractions. 
The features of Scala exemplify reusability and extensibility.

The standard implementation of Scala compiles to Java bytecode, executing in the Java run-time environment, on a \textit{Java Virtual Machine} (JVM). 
As a result, Scala is reliant on JVMs for compilation to low-level machine code and efficient execution. 
However, JVMs are predominantly optimized to improve the performance of Java programs. 
Furthermore, Scala program type information is significantly reduced during type erasure, a compilation phase which removes generic type information. 
Consequently, boxing, the operation of wrapping primitive values in an object, is unnecessarily introduced into the final program. 
Current state of the art techniques on eliminating boxing and achieving optimal object layouts at run-time, known as type specialization, rely on static program analysis.

\cleardoublepage

% A C K N O W L E D G E M E N T S
% -------------------------------

\begin{center}\textbf{Acknowledgements}\end{center}

I would like to thank all the little people who made this thesis possible.
\cleardoublepage

% D E D I C A T I O N
% -------------------

\begin{center}\textbf{Dedication}\end{center}

This is dedicated to the one I love.
\cleardoublepage

% T A B L E   O F   C O N T E N T S
% ---------------------------------
\renewcommand\contentsname{Table of Contents}
\tableofcontents
\cleardoublepage
\phantomsection    % allows hyperref to link to the correct page

% L I S T   O F   T A B L E S
% ---------------------------
\addcontentsline{toc}{chapter}{List of Tables}
\listoftables
\cleardoublepage
\phantomsection		% allows hyperref to link to the correct page

% L I S T   O F   F I G U R E S
% -----------------------------
\addcontentsline{toc}{chapter}{List of Figures}
\listoffigures
\cleardoublepage
\phantomsection		% allows hyperref to link to the correct page

% GLOSSARIES (Lists of definitions, abbreviations, symbols, etc. provided by the glossaries-extra package)
% -----------------------------
\printglossaries
\cleardoublepage
\phantomsection		% allows hyperref to link to the correct page

% Change page numbering back to Arabic numerals
\pagenumbering{arabic}

 

%----------------------------------------------------------------------
% MAIN BODY
%----------------------------------------------------------------------
% Because this is a short document, and to reduce the number of files
% needed for this template, the chapters are not separate
% documents as suggested above, but you get the idea. If they were
% separate documents, they would each start with the \chapter command, i.e, 
% do not contain \documentclass or \begin{document} and \end{document} commands.
%======================================================================
\chapter{Introduction}

\acrfull{jit} compilation has seen great success in implementing runtimes for objected-oriented programming languages.
It has effectively generated efficient machine code in the presence of virtual dispatch arising from \textit{subtype} polymorphism.
While a call site may statically have many possible call targets, JIT compilation can incorporate dynamic runtime information to optimize the most frequently invoked call targets speculatively.
These speculative optimizations often enable compiled code to be inlined, a critical transformation in the context of JIT compilation.
Inlining compiled code generates opportunities for many further optimizations.

Many object-oriented languages have since incorporated the notion of generic programming, otherwise known as \textit{parametric} polymorphism.
Parametric polymorphism enables programs to be more modular and reusable as functions and data structures behave identically\cite{tapl} regardless of the types of their inputs.
Implementations of generic programming often come at the expense of program complexity and performance.
Static compilers for object-oriented languages with parametric polymorphism must compromise when selecting an appropriate data representation for polymorphic data types and functions.
This trade-off comes down to more optimal data layouts at the expense of space or uniform data layouts, which are not optimal for every type at the expense of performance.

The selection of an optimal data representation, or \textit{specialization}, of a polymorphic data structure relies on information typically found in the type-rich source language of programming languages.
Representations must be consistent throughout the whole program as code that manipulates such data structures assume their representations to be consistent.
Consequently, the specialization problem is best suited to compilers with access to whole program information during compilation.
However, this is not the case for object-oriented languages such as Java and Scala, which statically generate a uniform data representation for their polymorphic definitions to guarantee consistency throughout the whole program. 
Additionally, static compilers do not have sufficient runtime information, which is critical in making favourable optimization decisions compared to JIT compilers.
On the other hand, JIT compilers are ill-suited to whole program optimizations as they are best at the dynamic optimization of small regions of a program.
Therefore, the problem of specialization falls between static compilation and JIT compilation.

This thesis introduces \textsc{TastyTruffle}, an interpreter and JIT compiler which incorporates rich source-level type information with speculative optimizations to specialize data representations for the Scala programming language.
\textsc{TastyTruffle} is implemented in Truffle, a framework that simplifies the implementation of a JIT compiler for a source language by implementing an interpreter for that language. 
Our source language is the \acrfull{tasty} serialization format emitted by the Scala 3 compiler.
TASTy is an abstract syntax tree format emitted after parsing and type checking Scala programs.
By using TASTy, a suitable source language, we can access source-level type information without having to parse and type check a Scala source program.

The contributions of this thesis are as follows: 
\begin{enumerate}
	\item The implementation of an interpreter for the TASTy format using Truffle and the necessary transformations to make a TASTy program executable. TASTy is high-level uncanonical representation of Scala not suitable for execution; Non-trivial transformations must be applied to a TASTy program before execution.
	In contrast, Java bytecode of compiled Scala programs is readily available for execution on any Java virtual machine.
	\item The extension of interpreter to support specialized data representations of generic types.
	These specialized data representations are created using concrete type arguments that generic types are instantiated with.
	with generic data structures.
	\item The evaluation of the interpreter on simple and realistic programs that present a challenge to existing state-of-the-art techniques.
\end{enumerate}

\newpage

\section{Thesis Organization}

We describe the layout of the remainder of this thesis.
Chapter 2 provides an overview of the many intermediate representations of Scala from compilation to execution.
It explores the advantages and drawbacks of each intermediate representation concerning specialization.
Chapter 3 details the implementation of \textsc{TastyTruffle}.
It covers the translation of TASTy into a more suitable IR for execution in an interpreter where each polymorphic data structure has a uniform representation.
The chapter then provides extensions to the interpreter to support the just-in-time specialization of polymorphic data structures.
Chapter 4 evaluates the interpreter with and without extensions for dynamic specialization on simple but realistic data structures.
The chapter provides the performance of these evaluated data structures in the context of the standard implementation of Scala with the underlying JIT compiler of our interpreter without any augmentation.
Chapter 5 explores related work in various implementations of parametric polymorphism and other Truffle interpreters.
Chapter 6 discusses possible extensions to \textsc{TastyTruffle} to better integrate source-level type semantics with JIT compilation.
Chapter 7 concludes the thesis.
%======================================================================
\chapter{Background}

In this chapter, we will provide an introduction to the Scala programming language. 
We will showcase a running example that we will use for the remainder of this thesis which exhibits features commonly present in Scala programs. 
We will describe \acrfull{tasty}, an intermediate storage format used for separate compilation\cite{???} of Scala programs. 
We will introduce a critical transformation, type erasure, which alters Scala programs so that they may executable on their default platform the \acrfull{jvm}. 
We will detail GraalVM \acrfull{jit} compiler infrastructure, an alternative JVM implementation which we use to implement a runtime for Scala in this thesis.

\section{Scala}

Scala\cite{scala:overview} is an objected-oriented, generic and statically typed programming language.
Scala uses a \textit{pure} object-oriected programming model\cite{smalltalk:design} and addresses many of the shortcomings\cite{go4:design-patterns} in other object-oriented programming languages.
Scala can be still considered \textit{Java-like} because of the interoperability between Java and Scala programs.
Programs in Scala may contain generic definitions, allowing Scala programs to be composable and reusable\cite{scala:origins}.
While these features offer abstractions which facilitate the design of increasingly complex programs, there are significant challenges with their implementation.
In the subsequent sections of this chapter, we will describe the challenges implementing these paradigms when manifested in the various intermediate representations of Scala.
We first begin with an explanation of the relevant programming paradigms present in Scala:

\begin{description}
	\item[Object-oriented] 
	Every value in Scala is an object and every operation is method invocation on an object. 
	Every object in Scala is an instance of a \textit{class} and their type is determined by its class.
	Classes\cite{simula:classes} are a mechanism for defining state and behaviour for a group of objects.	
	
	\item[Generic] 
	Classes in Scala may contain \textit{type parameters} and such classes can be considered \textit{polymorphic}\cite{strachey:fundamental-concepts}.
	Polymorphic classes may define behavior independent of their data, allowing them to be reused extensively for multiple types of data.
	In this thesis, we will interchangeably use the term \textit{parametric polymorphism} to refer to generics.
	
	\item[Statically typed] 
	Static typing is a discipline where the type information about a program is known \textit{before} it is executed.
	In order for a Scala program to compile successfully, it must be \textit{well-typed}.
	For our purposes, computation should always produce a value which has a type matching the type declared by the programmer to be considered well-typed.
	Classes are the primary syntactical mechanism for declaring types in Scala. 
	The properties of classes such as state, in the form of fields, and behaviour, in the form of methods, must be well-typed.
	Similarly, the uses of these properties in other classes must also be well-typed. 
\end{description}

\section{Case Study: A List in Scala}

In this section, we will introduce the running example that will be used for the remainder of this thesis and our motivations for its selection.
Figures \ref{example:list-def}, \ref{example:cons-impl} and \ref{example:nil-impl} contain an abstract singly-linked list class and its two concrete subclass implementations. 
This set of \scalainline{List} implementations represent probable real-world use cases as they are a scaled down and simplified version of the list implementation present in the Scala collections library.
The \scalainline{List} definition from the collections library is available by default to all Scala programs.

\begin{figure}[!htb]
	\begin{minted}{scala}
	abstract class List[+T] {
		def head: T
		def tail: List[T]
		def length: Int
		def isEmpty: Boolean = length == 0
		def contains[T1 >: T](elem: T1): Boolean
	}
	\end{minted}
	\caption{Definition of \texttt{List} class}
	\label{example:list-def}
\end{figure}

Figure \ref{example:list-def} is an example which showcases the paradigms discussed in the previous section that are also commonly present real-world Scala programs.
Implementations which extend the abstract \scalainline{List} class exhibit the object-oriented property of \textit{inheritance}.
The \scalainline{List} class contains a mixture of polymorphic and non-polymorphic methods to showcase type specialization
The \scalainline{head} method is class-polymorphic in that its type is derived from a class parameter and becomes specialized when the class is specialized.
The \scalainline{contains} method is method-polymorphic and must be specialized after the class is specialized.

\begin{figure}[!htb]
	\begin{minted}{scala}
	case class Cons[+T](head: T, tail: List[T]) extends List[T] {
		override def length: Int = 1 + tail.length
		
		override def contains[T1 >: T](elem: T1): Boolean = {
			var these: List[T] = this
			while (!these.isEmpty) 
				if (these.head == elem) return true
				else these = these.tail
			false
		}
			
		override def hashCode(): Int = {
			var these: List[T] = this
			var hashCode: Int = 0
			while (!these.isEmpty) {
				val headHash = these.head.## // Compute hashcode
				if (these.tail.isEmpty) hashCode = hashCode | headHash
				else hashCode = hashCode | headHash >> 8
				these = these.tail
			}
			hashCode
		}
	}
	\end{minted}
	\caption{Implementation of \texttt{Cons} class}
	\label{example:cons-impl}
\end{figure}

Figure \ref{example:cons-impl} contains the implementation of a list node.
The \scalainline{Cons} implementation contains two polymorphic fields, \scalainline{head} and \scalainline{tail}.
For specialization, how the \scalainline{head} field fits into the storage layout of a \scalainline{Cons} instance may differ between a \scalainline{Cons[Int]} and a \scalainline{Cons[String]}.
On the other hand, the \scalainline{tail} field does not have to differ between instances of \scalainline{Cons[Int]} and \scalainline{Cons[String]}.

\begin{figure}[!htb]
	\begin{minted}{scala}
	case object Nil extends List[Nothing] {
		override def head: Nothing = throw new NoSuchElementException("head of empty list")
		override def tail: Nothing = throw new UnsupportedOperationException("tail of empty list")
		override def length: Int = 0
		override def contains[T1 >: Nothing](elem: T1): Boolean = false
		override def hashCode(): Int = 0
	}
	\end{minted}
	\caption{Implementation of \texttt{Nil} class}
	\label{example:nil-impl}
\end{figure}

Figure \ref{example:nil-impl} contains the implementation of the empty list. 
We provide the implementation of this class for completeness.


\section{Typed Abstract Syntax Trees}

An \acrfull{ir} is a structural abstraction representing a program during compilation or execution. 
Intermediate representations are more suitable for reasoning about a program than program source code. 
\acrshort{ir} can be used for compilation\cite{llvm}, optimization\cite{llvm}\cite{ssa}, or execution\cite{java:vm-spec}\cite{clr:spec}.

\acrfull{tasty} is a high-level \acrfull{ir} which is produced and emitted after the type checking phase (also called the typer) of the Scala compiler (see appendix \ref{appendix:dotty-phases}).
Figure \ref{system:tasty} gives an overview of TASTy generation in the context of the Scala compilation pipeline, note that TASTy is only generated for Scala program sources.
TASTy is a well-typed variation of an \acrfull{ast}.
Abstract syntax trees are a commonly used intermediate representation which resemble the program source representation.
TASty can be considered a \textit{complete} IR of a Scala program before compilation, unlike the other intermediate representations we will examine throughout this thesis.
A complete IR is able to capture all information of the original Scala source program.
We will expand on why complete intermediate represenations are significant in section \ref{background:type-erasure}.

\begin{figure}[H]
	\centering
	\includegraphics[width=0.4\textwidth]{figures/scala-pipeline.png}
	\caption{TASTy in the context of the Scala compilation pipeline.}
	\label{system:tasty}
\end{figure}

The full TASTy IR can represent all Scala programs.
The Truffle interpreter in this thesis supports a subset sufficient to express the the programs given in figures \ref{example:list-def} and \ref{example:list-impl}.
The TASTy trees used in this thesis can be divided into categories, definitions, terms, and types. 
We give the pseudo implementations of these trees in figures: \ref{tasty:defs}, \ref{tasty:terms}, and \ref{tasty:types}.

\subsection{Definitions}

\begin{figure}[!htb]
	\begin{minted}{scala}
	// Tree representing code written in the source
	trait Tree {
		def symbol: Symbol
	}                         
	trait Statement extends Tree       // Tree representing a statement in the source code
	trait Definition extends Statement // Tree representing a definition in the source code.
		
	// Tree representing a class definition.
	case class ClassDef(
		name:        String,
		constructor: DefDef, 
		parents:     List[Tree], 
		self:        Option[ValDef], 
		body:        List[Statement]
	) extends Definition
	// Tree representing a method definition in the source code
	case class DefDef(
		name:      String, 
		params:    List[ParamClause], 
		returnTpt: TypeTree, 
		rhs:       Option[Term]
	) extends Definition
	// Tree representing a value definition in the source code.
	case class ValDef(name: String, tpt: TypeTree, rhs: Option[Term]) extends Definition
	// Tree representing a type (parameter or member) definition in the source code
	case class TypeDef(name: String, rhs: Tree) extends Definition
	\end{minted} 
	\caption{Pseudocode class definitions for a subset of TASTy trees.}
	\label{tasty:defs}
\end{figure}

A Scala program consists of top level class definition which themselves contain statements.
Statements either represent a declaration inside a class, such as method definitions, or executable code (or terms), which we discuss in section \ref{section:tasty:terms}.
Figure \ref{tasty:defs} provides the pseudo implementations of all definitions in our subset of TASTy.
Every tree has a symbol, which is a unique reference to a definition.
For the use cases in this thesis, most definitions can be translated and be represented by a corresponding implementation in Truffle.
A \scalainline{ClassDef} represents a top level class definition.
A \scalainline{DefDef} tree is the definition of a method inside a class definition.

A \scalainline{ValDef} tree is a context-dependent definition which represents different value definition semantics depending on the tree it is defined in.
A top level \scalainline{ValDef}, that is a \scalainline{ValDef} with no parent represents the \scalainline{object} abstraction in Scala.
The \scalainline{object} abstraction is commonly used to represent the \textit{Singleton} pattern\cite{go4:design-patterns} or as a class-like interface to define static methods.
Consider the \scalainline{Nil} class given in figure \ref{example:nil-impl}, a simplified TASTy-like equivalent would resemble the following:

\begin{figure}[!htb]
	\begin{minted}{scala}
	val Nil = new Nil$
	class Nil$ extends List[Nothing] { ... }
	\end{minted} 
	\caption{Simplified implementation of the \scalainline{object Nil}}
	\label{example:decomp-object}
\end{figure}

A \scalainline{ValDef} tree defined in the \scalainline{body} of a \scalainline{ClassDef} tree represent field definitions.
A \scalainline{ValDef} tree defined in the \scalainline{TermParam} section of a \scalainline{DefDef} tree represent parameter definitions of the method.
A \scalainline{ValDef} tree defined among the statements in a \scalainline{Block} tree is a local variable definition limited to the scope of the block.

Similarly, \scalainline{TypeDef} trees refer to different kinds of definitions depending on their definition site.
A \scalainline{TypeDef} in the body of a \scalainline{ClassDef} refers to a polymorphic class type parameter in our subset of TASTy.
When a \scalainline{TypeDef} is located in the \scalainline{TypeParam} section a \scalainline{DefDef} tree, it refers to a polymorphic method type parameter.
The trees defined here can be used to represent more complex object-oriented and functional abstractions such as nested classes or closures, but they are beyond the scope of this thesis.

Figure \ref{tasty:list} is the TASTy structure of the \scalainline{List} class given in figure \ref{example:list-def}. 
Recall that \scalainline{ClassDef} trees have four structural components, the constructor, the list of parent class definitions, the self type, and the body of the definition.
In this thesis, we will not discuss the self type as it is an abstraction for composition\cite{gilad:mixins}\cite{scala:calculus} and is not relevant for execution.
The list of parents in a class definition in our subset of TASTy is always a singleton.
Note that while the abstract \scalainline{List} class did not explicitly declare a constructor, the compiler autogenerates and inserts the appropriate constructor implementation before emitting TASTy.
Since \scalainline{List} is polymorphic, it contains an inner type definition of its sole type parameter.
This distinction is what makes TASTy a complete IR when compared to the other intermediate representations we will describe later in this chapter.

\begin{figure}[!htb]
	\begin{minted}{scala}
	ClassDef(
		// name 
		"List",
		// constructor
		DefDef("<init>", List(TypeParams(TypeDef("T", TypeBoundsTree(_, _)), TermParams(Nil)), _, None)),
		// parents
		List(Apply(Select(New(_, "<init>"), Nil))),
		// self
		None,
		// body
		List(
			TypeDef("T", TypeBoundsTree(_, _)),
			DefDef("head", Nil, TypeIdent("T"), None),
			DefDef("tail", Nil,Applied(TypeIdent("List"), List(TypeIdent("T"))),None),
			DefDef("length", Nil, TypeIdent("Int"), None),
			DefDef("isEmpty", Nil, TypeIdent("Boolean"), None),
			DefDef(
				"contains",
				List(
					TypeParams(TypeDef("T1", TypeBoundsTree(TypeIdent("T"), _))),
					TermParams(ValDef("elem", TypeIdent("T1"), None))
				),
				TypeIdent("Boolean"),
				None
			)
		)
	)
	\end{minted} 
	\caption{Tree structure for the definition of \texttt{List} . For brevity, we use \textbf{\texttt{\_}} to represent inferred\cite{ml:type-inference} type trees by the compiler.}
	\label{tasty:list}
\end{figure}

Similarly, \scalainline{DefDef} trees also retain their polymorphic properties.
The parameters section of a \scalainline{DefDef} tree is split into two halves.
The type parameter section preserves any polymorphic type parameters in the method definition.
The term parameter section contains the normal value parameters found in a method.
Term parameters may have types which are derived from the type parameter section.

\subsection{Terms}
\label{section:tasty:terms}

\begin{figure}[!htb]
	\begin{minted}{scala}
	// Tree representing an expression in the source code
	trait Term extends Statement {
		def tpe: Type
	}
	// Tree representing a reference to definition      
	trait Ref extends Term             
	
	// Tree representing an assignment lhs = rhs in the source code
	case class Assign(lhs: Term, rhs: Term) extends Term
	// Tree representing new in the source code
	case class New(tpt: TypeTree) extends Term
	// Tree representing a block `{ ... }` in the source code
	case class Block(statements: List[Statement], expr: Term) extends Term
	// Tree representing a while loop
	case class While(cond: Term, body: Term) extends Term
	// Tree representing an if/then/else if (...) ... else ... in the source code
	case class If(cond: Term, thenp: Term, elsep: Term) extends Term
	// Tree representing a return in the source code
	case class Return(expr: Term, from: Symbol) extends Term
	// Tree representing a selection of definition with a given name on a given prefix
	case class Select(qualifier: Term, selector: String) extends Term 
	// Tree representing an application of arguments.
	case class Apply(applicator: Term, arguments: List[Term]) extends Term
	// Tree representing an application of type arguments
	case class TypeApply(fun: Term, args: List[TypeTree]) extends Term
	// Tree representing a reference to definition with a given name
	case class Ident(name: String) extends Ref 
	// Tree representing constant value
	case class Constant(value: Int | ... | String) extends Term 
	\end{minted} 
	\caption{Pseudocode class definitions for a subset of TASTy trees.}
	\label{tasty:terms}
\end{figure}

Figure \ref{tasty:terms} gives the implementation for terms in our subset of TASTy.
Terms represent executable atoms of code which return values.
Terms can be considered analogous to expressions from the abstract syntax trees commonly used for other imperative programming languages.
Our term tree subset of TASTy represents a basic language with support for simple imperative programming with control flow constructs such as branching and loops.
A basic set of object-oriented features are also encapsulated in the tree definitions given above.
The set of object-oriented features include object creation, instance method invocation, and instance field access.
This subset of TASTY is sufficient to represent the creation of polymorphic classes as well as the invocation of polymorphic methods to showcase the examples described in this thesis.

Terms in TASTy also retain their types after type checking by the Scala compiler.
A type for a term describes the type of the value produced by the term.
Terms with no children, such as \scalainline{Ident} trees, are \textit{explicit} typed.
Childrenless terms have their type information encoded in a TASTy file.  
For terms with children, their types are derived from those of their children trees.
Type information for non-leaf term trees is regenerated from term leaves when a TASTy file is read.
In essence, types `flow' upwards from leaf nodes in TASTy to their parent terms until the root term.
The interpreter described in this thesis intreprets a tree where the types for all trees have been regenerated.
We will describe types in detail in the following section.

\subsection{Types and Type Trees}

TASTy encodes Scala programs with two kinds of type information, type trees and types.
Type trees are a subset of trees which represent types as they are declared in Scala source code.
On the other hand, types are the canonical representation of type trees after type checking in the Scala compiler.
Multiple type trees may denote the same underlying type.

\begin{figure}[!htb]
	\begin{minted}{scala}
	// Type tree representing a type written in the source
	trait TypeTree extends Tree {
		def tpe: Type
	}
	
	// Type tree representing a reference to definition with a given name
	case class TypeIdent(name: String) extends TypeTree 
	// Type tree representing a type application
	case class Applied(tpt: TypeTree, args: List[TypeTree | TypeBoundsTree]) extends TypeTree
	// Type tree representing a type bound written in the source
	case class TypeBoundsTree(lo: TypeTree, hi: TypeTree) extends TypeTree
	\end{minted} 
	\caption{Pseudocode class definitions for a subset of TASTy type trees.}
	\label{tasty:type-trees}
\end{figure}

Figure \ref{tasty:type-trees} gives the subset of type trees which we will use in this thesis.
For our purposes, there are only three ways to refer to types.
A \scalainline{TypeIdent} type tree is a reference to a type which is a \scalainline{ClassDef}.
An \scalainline{Applied} type tree represents a type constructor, which accepts type arguments and produces a new type.
For example, the type \scalainline{Cons[T]} would be represented as an applied type tree, where \scalainline{Cons} would the constructor and \scalainline{T} would be the type argument.
A \scalainline{TypeBounds} tree represents the type expression \scalainline{Lo <: T <: Hi}, a constraint where \scalainline{T} must be a subtype of type \scalainline{Hi} and supertype of type \scalainline{Lo}.
Type bounds are typically used to represent declared type parameter constraints, otherwise known as \textit{bounded quantification}\cite{systemF:subtyping}, in polymorphic classes or polymorphic methods.
However, type bounds are also inserted by the Typer because type parameters in TASTy are universally contraints.
A type parameter \scalainline{T} is expanded to \scalainline{Nothing <: T <: Any}, that is the type parameter \scalainline{T} must be a subtype of \scalainline{Any} and a supertype of \scalainline{Nothing}.
In the context of this thesis, we can use subtype to mean \textit{subclass of} and supertype to mean \textit{superclass of}.
Practically, this means the type parameter \scalainline{T} has no constraints since \scalainline{Any} is the super type of all types and \scalainline{Nothing} is the subtype of all types.

\begin{figure}[!htb]
	\begin{minted}{scala}
	trait Type                           // A type, type constructors, type bounds
	trait NamedType extends Type         // Type of a reference to a type or term symbol
	case class TypeRef extends NamedType // Type of a reference to a type symbol
	case class AppliedType extends Type  // A higher kinded type applied to some types T[U]
	case class TypeBounds extends Type   // Type bounds
	\end{minted} 
	\caption{Pseudocode class definitions for a subset of TASTy type trees.}
	\label{tasty:types}
\end{figure}

Figure \ref{tasty:types} is set of types used in our subset of TASTy.
In most cases in our subset of TASTy, the type trees have a corresponding type of the same name.
However, the \scalainline{NamedType} does not appear in type trees as they are predominantly used to type terms.
The \scalainline{TypeRef} type is a reference to a \scalainline{ClassDef} tree or a type parameter \scalainline{TypeDef}.

In the Scala compilation pipeline, TASTy is eventually simplified and transformed by the Scala compiler to produce Java bytecode. 
In chapter \ref{chapter:implementation}, We will go over each tree before such transformations and their relevance for execution in our interpreter .

\section{Java Bytecode}

Java bytecode is a portable and compact intermediate language and instruction set used by the Java Virtual Machine to execute programs.
Java bytecode can be considered similar to an assembly language, where programs are represented as sequences of atomic instructions which manipulate a stack or registers.
The type system in Java bytecode can describe primitive values such as \javainline{int} and references to objects such as \javainline{String}.
As bytecode is intended to be simple for execution, it is not possible to represent polymorphic programs fully in Java bytecode.

Types in TASTy are not immediately compatible with types available in Java bytecode.
Scala's type semantics must be eliminated from programs by the compiler before Java bytecode of the program can be emitted.
The resulting Java bytecode is considered an \textit{incomplete} IR of Scala source programs, as the type information found in the program source or inferred from compilation is no longer present.
This becomes a particular drawback for executing Scala programs on the JVM because speculative optimizations are unable to incorporate source level semantics.

\begin{figure}[!htb]
	\begin{minted}{scala}
	aload_0
 	astore_2
	aload_2
	invokevirtual #44 // List.isEmpty:()Z
	ifne          30
	aload_2
	invokevirtual #46 // List.head:()Ljava/lang/Object;
	aload_1
	invokestatic  #52 // Method scala/runtime/BoxesRunTime.equals:(Ljava/lang/Object;Ljava/lang/Object;)Z
	ifeq          22
	iconst_1
	ireturn
	aload_2
	invokevirtual #53 // List.tail:()LList;
	astore_2
	goto          2
	iconst_0
	ireturn
	\end{minted}
	\caption{Java bytecode of \texttt{Cons.contains}}
	\label{example:contains-bytecode}
\end{figure}

Figure \ref{example:contains-bytecode} is the Java bytecode of the \scalainline{contains} defined at line 4 in figure \ref{example:cons-impl}.
Typical control flow elements of Scala programs such as if terms and while terms have been converted into branch and jump instructions.
Notice that there are no polymorphic type parameters in the description of classes nor in the invocation of polymorphic methods present in the bytecode.
In particular, notice the equality comparison in line 7 of figure \ref{example:cons-impl} is actually a method invocation (instruction 14 in figure \ref{example:contains-bytecode}).
As the Scala compiler is unable to determine the type of a polymorphic type parameter during complilation time, it is unable to select a Java bytecode instruction which implements polymorphic comparison.
Instead, a bridge method part of the Scala standard library is responsible for handling polymorphic operations which operate on both reference and primitive types during runtime.
In the next section, we describe the process which transforms Scala programs to a reprensentation amenable for Java bytecode generation and the necessary additional runtime overhead associated with this transformation.

\section{Type Erasure}
\label{background:type-erasure}

Type erasure\cite{java:generics} is a transformation which converts polymorphic classes and methods in Scala to monomorphic classes and methods. 
This conversion is necessary because the JVM does not support polymorphic classes during runtime.
Erasure ensures that any given polymorphic class and method has a single representation in practice.
Type erasure is a crucial part of Scala compilation that renders TASTy incomplete.
Figure \ref{example:erase-cons} shows the \scalainline{Cons} class after type erasure.

\begin{figure}[!htb]
	\begin{minted}{scala}
	case class Cons(head: Any, tail: List) extends List {
		override def length: Int = 1 + tail.length
			
		override def contains(elem: Any): Boolean = {
			var these: List = this
			while (!these.isEmpty) 
			if (these.head == elem) return true
			else these = these.tail
			false
		}
			
		override def hashCode(): Int = {
			var these: List = this
			var hashCode: Int = 0
			while (!these.isEmpty) {
				val headHash = these.head.##
				if (these.tail.isEmpty) hashCode ||= headHash
				else hashCode |= headHash >> 8
				these = these.tail	
			}
			hashCode
		}
	}		
	\end{minted}
	\caption{\scalainline{Cons} class after type erasure}
	\label{example:erase-cons}
\end{figure}

The polymorphic \scalainline{Cons} class has all type parameters in its class definition \textit{erased} and replaced by the \scalainline{Any} type.
The \scalainline{Any} type is a Scala platform-independent\cite{scala:overview} abstract type representing the supertype of primitive and reference types.
In Java bytecode, the {Any} type is compiled to the \scalainline{Object} type, the supertype of all reference types on the JVM.

While type erasure simplifies classes for runtime, the Scala compiler must resolve the incompatibility of operations between primitives types and reference types on the JVM\cite{java:vm-spec}.
In order for primitive types to have a uniform representation compatible with reference types, primitive types are encapsulated into corresponding boxed classes whose objects are passed by reference.
For example, \javainline{java.lang.Integer} is a class with an \scalainline{Int} field.
In a polymorphic context in which a type variable has been replaced by the reference type \javainline{Object}, an \scalainline{Int} value is not passed directly, but by reference to an object of class \javainline{Integer} that contains the primitive value.
The set of operations introduced by the compiler whenever a primitive value is accessed under a polymorphic context is known as \textit{autoboxing}\cite{java:autoboxing}. 
Autoboxing can be divided into two operations.
\textit{Boxing} occurs when a primitive value must be used where a polymorphic value is expected.
\textit{Unboxing} occurs when a polymorphic value must be used where a primitive value is expected.
Figure \ref{example:autoboxing} shows a simple example of inserted autoboxing operations when using the polymorphic \scalainline{Cons} class after type erasure.

\begin{figure}[!htb]
	\begin{minted}{scala}
	// Before type erasure 	
	val lst: List[Int] = Cons(1, Nil)
	val head: Int = lst.head
	// After type erasure
	val lst: List = Cons(box(1), Nil)
	val head: Int = unbox(lst.head) 
	\end{minted}
	\caption{Example of autoboxing introduced for a list}
	\label{example:autoboxing}
\end{figure}

The \scalainline{head0} field inside the \scalainline{Cons} class after erasure is no longer polymorphic and instead has the type \scalainline{Any}. 
The integer value of \scalainline{1} which is passed into the class constructor for the list is boxed and the primitive value is wrapped as an instance of its boxed class.
Similarly, when the \scalainline{head0} field of the instance is read and stored into a local variable, an unboxing operation occurs which extracts the primitive value out of its wrapper instance.
In the Scala collections library, a set of commonly used polymorphic data structures, autoboxing operations are frequent and necessary.
The computational overheads of autoboxing operations on programs which make substantial use of polymorphic collections, especially the Scala standard library, is significant\cite{scala:collections-optimization}.
The elimination of this overhead through optimizing autoboxing operations is one of the central goals of this thesis.
In addition to this direct overhead, autoboxing is a significant indirect source of overhead which makes the analysis of programs using primitive values in a polymorphic context and thus inhibits many significant compiler optimizations.

\section{GraalVM}

GraalVM\cite{java:graalvm} is an implementation of a JVM.
Traditionally, the JVM is responsible for the majority of the performance optimizations in Java programs\cite{java:hotspot} through \acrfull{jit} compilation.
JIT compilation is an adaptive optimization which occurs during program execution.
JIT compilation is concerned with optimizing and eliminating \textit{hotspots} or portions of the program which are executed most frequently.
JIT compilers\cite{java:sablevm}\cite{java:jikesrvm} employ a range of \textit{speculative} techniques to transform the program under optimization.
Speculative optimizations use information collected during program execution, otherwise known as \textit{profiling}. 
Assumptions are then made about gathered profiling data in order to generate high-performance native machine code.
A key aspect of speculative optimizations using assumptions is that optimizations may be undone when their underlying assumptions are violated.
This enables the JIT compiler to optimize programs without the need to statically prove assumptions hold in every execution path.

While other implementations of Java virtual machines were designed specifically for Java, GraalVM was designed from the onset to be \textit{language-independent}.
GraalVM can be divided into two major components of interest. 
The first is \textit{Graal}, a language-agnostic JIT compilation infrastructure which handles speculative optimizations and generation of high-performance machine code.
The second is \textit{Truffle}, a framework for translating the semantics of a source language, also called a \textit{guest language}, to take advantage of the Graal infrastrucure.

\begin{figure}[!htb]
	\centering
	\includegraphics[width=0.5\textwidth]{figures/graalvm-pipeline.png}
	\caption{GraalVM overview\cite{graalvm:ir}.}
\end{figure}

This thesis makes substantial of both components of GraalVM to create a runtime for Scala programs using TASTy.
The runtime is able to incorporate source level information for speculative optimizations.


\subsection{Graal}

GraalVM incorporates an existing implementation of a JVM\cite{java:hotspot} for the actual execution of programs.
Graal is \textit{only} the general-purpose just-in-time compilation infrastructure which optimizes the programs to be executed.
Graal is general-purpose in that it conducts analysis and optimization on the same intermediate representation, \textit{Graal IR}, regardless of the original source language.
Notably, most implementations of a source language utilizing GraalVM have an implementation in Truffle, Java is an exceptional case.
In addition to a Truffle interpreter for Java bytecode\cite{graalvm:espresso}, there is a direct translator for Java programs in GraalVM which parses Java bytecode into Graal IR.

Graal IR\cite{graalvm:ir}\footnote{Given the number of intermediate representations introduced thus far, we promise this is the last one} is an IR which is suitable for speculative optimizations while still retaining information from the Truffle guest language AST.
Graal IR is based on the \textit{sea of nodes} concept\cite{click:sea-of-nodes} and satisfies the \textit{static single-assignment}\cite{ssa} property.
A sea of nodes is an abstraction based on a directed graph structure which relates the control flow graph\cite{allen:ctrl-flow-analysis} of a program to its data flow graph\cite{allen:data-flow-analysis}.
An intermediate representation is in single-static assignment form when each variable is declared once and every use of a variable occurs immediately after its declaration\cite{johnson:use-def-chains}.

GraalIR enables Graal to speculatively compile only the \textit{hot} branches\cite{graalvm:speculative-ir}, or branches that are most frequently taken, in the control flow portion of the IR and their transitive data dependencies.
When a compiled program violates any of its underlying assumptions, execution is \textit{deoptimized}\cite{self:deoptimization} and the program resumes execution in its uncompiled format.
Deoptimization occurs when the compiled program is no longer considered stable and therefore is invalid.
Graal automatically inserts \textit{guard nodes} into the IR, which are conditional checks which validate that speculative assumptions used to compile the program still hold.
Deoptimization is part of an execution loop between Graal and Truffle which allows GraalVM to aggressively adapt and speculate to find the best optimization in a dynamic execution environment.

\smartdiagramset{
	text width=2.75cm,
	uniform color list=white for 3 items,
	uniform arrow color=true,
	arrow color=black,
}
\begin{figure}[!htb]
	\centering
	\scalebox{0.7}{
		\smartdiagram[circular diagram:clockwise]{
			Node rewriting,
			Partial evaluation,
			Deoptimization
	}}
	\caption{Adaptive optimization loop of GraalVM}
	\label{diagram:graal-loop}
\end{figure}

\subsection{Truffle}

Truffle is a framework for implementing an interpreter embedded into GraalVM.
Truffle differs signficantly from other implementations of interpreters.
Interpreters can usually be divided into two subsets: tree interpreters and bytecode interpreters.
Tree interpreters transform program source into an abstract syntax tree which is then executed in post-order, children nodes are executed before their parents.
Abstract syntax tree interpretation has the added benefit of executing an intermediate representation which is quite close to the program source representation and is therefore more amenable to program optimization.
In contrast, bytecode interpreters such as the JVM, execute a vastly simplified representation of programs.
While interpreters of bytecode programs tend to be faster than their tree counterparts, the absence of detailed source information such as types often makes program optimization difficult.
The challenge of efficiently executing bytecode while retaining the ability to optimize them effectively using source program information is particularly difficult for Scala on the JVM. 

\begin{figure}[!htb]
	\begin{minted}{scala}
	abstract class EqualsNode extends BinaryOpNode {
		@Specialization
		def equalsInt(lhs: Int, rhs: Int): Boolean = lhs == rhs
		
		@Specialization(replaces="equalsInt")
		def equals(lhs: Any, rhs: Any): Boolean = if (lhs == null) rhs == null else lhs.equals(rhs)
	}
	\end{minted}
	\caption{Pseudocode for a Truffle node implementation of an equality which supports node rewriting.}
	\label{example:node-rewriting}
\end{figure}

Truffle is an atypical tree interpreter in that it combines the definition, execution, and optimization of an abstract syntax tree structure into a single abstraction.
While the structure of input programs in other interpreters is independent of the implemenation of the interpreter, a Truffle interpreter is integrated into the structure of its input.
By defining execution semantics inside the abstract syntax tree to be executed, an interpreter is essentially derived from the implementation of its input tree structure.
The execution semantics of the AST are additionally augmented with the Truffle \acrfull{dsl}, which allows such trees to be \textit{self-optimizing}.
The Truffle DSL is a mechanism to allow a \textit{guest language} to embed semantics into a Truffle AST for optimization.
A guest language is a set of semantics, most commonly a programming language, which is described by a Truffle AST.
In this thesis, the guest language which our Truffle AST encodes and executes is TASTy (which represents Scala).

\begin{figure}[!htb]
\begin{minted}{java}
	@GeneratedBy(AnyEqNode.class)
	public final class AnyEqNodeGen extends AnyEqNode {
		@Child
		private TermNode lhs_;
		@Child
		private TermNode rhs_;
		@CompilationFinal
		private int state_0_;
		
		private AnyEqNodeGen(TermNode lhs, TermNode rhs) {
			this.lhs_ = lhs;
			this.rhs_ = rhs;
		}
		
		public Object execute(VirtualFrame frame) {
			int state = this.state_0_;
			return (state & 2) == 0 && state != 0 ? 
				this.execute_int_int0(state, frame) : 
				this.execute_generic1(state, frame);
		}
\end{minted}
\end{figure}

During execution of the AST, profiling information collected from the interpreter is used to drive \textit{node rewriting}.
While Graal is language-agnostic, Truffle is able to exploit guest language semantics for dynamic optimizations.
This process of replacing nodes in the AST with better, specialized guest language counterparts in Truffle is called node rewriting.
Node rewriting makes Truffle abstract syntax trees self-optimizing and serves two purposes.
The first is to dynamically incorporate guest language semantics into the executing program.
The second is to augment the AST for more efficient JIT compilation.
The nature of compiler optimizations require that programs are incrementally simplified in order to be optimized.
While such types of optimizations are widely applicable to many languages using the JVM, node rewriting is a high-level language-specific optimization which occurs \textit{before} such simplifications.

Figure \ref{example:node-rewriting} demonstrates an example of a node which can be rewritten.
The node declares semantics of the equality operation between integers and values of type \scalainline{Any}.
This equality node has semantics for every type because the \scalainline{Any} type is the super type of all types in Scala .
A Truffle node which can be rewritten starts off in the uninitialized state.
When both the left and right hand side operands are integers, the node is rewritten to \javainline{equalsInt} state.
When arguments of any other combination of types are detected, either in the uninitialized state or the \javainline{equalsInt} state, the node is rewritten to the \javainline{equals} state.

After node rewriting, Graal JIT compiles Truffle ASTs into native machine code using \textit{partial evaluation}.
Partial evaluation is a program optimization technique for specializing a program (code) for a given input (data)\cite{futamura:partial-eval}.
In the context of Truffle, this means specializing an AST node (code) based on the values (or types of values) produced by their children nodes (data)\cite{truffle:partial-eval}.
We can say that the partial evaluation of an AST  will produce an AST which is \textit{specialized} for a particular set of values, or more commonly in our case, a particular set of types.
If a frequently executed Truffle AST cannot be rewritten further, it is considered \textit{stable} for JIT compilation into native machine code.  
The sequence of optimizations given in figure \ref{diagram:graal-loop}, node rewriting, partial evaluation into machine code, and deoptimization is the advantage that a TASTy Truffle interpreter has over the traditional JVM bytecode interpreter for Scala.
Truffle allows for the incorporation of source-level type information into the just-in-time compilation loop.
In this thesis, we will focus heavily on using node rewrites in the execution of TASTy with type information to augment JIT compilation.


%======================================================================
\chapter{Implementation}

\section{TastyTruffle Intermediate Representation}

Scala programs in \acrshort{tasty} format are unsuitable for execution in a Truffle interpreter. Programs in must be parsed and transformed into an executable representation in \textsc{TastyTruffle}. As TASTy represents a Scala program close to its equivalent source representation, the transformation of TASTy IR into TastyTruffle IR is not isomorphic. 
TastyTruffle IR represents a canonicalized executable intermediate representation which can be specialized on demand. 

The following sections will introduce the nodes in TastyTruffle IR and how they are derived from Scala source and TASTy.

TODO: insert TASTy tree diagrams.

\subsection{Root Node}

\subsection{Read and Write Nodes}

The retrieval and storage of values in TastyTruffle IR can be divided into the 

access or assignment to a local variable, access or assignment to a 

\subsubsection{\mintinline{scala}|(x: T)|}

\subsection{Control Flow Nodes}

\subsection{Call Nodes}

\subsection{Type Nodes}

\subsection{Allocation Nodes}

\subsubsection{\mintinline{scala}|new Foo|}

\subsubsection{\mintinline{scala}|new Array[Int]|}

\subsubsection{\mintinline{scala}|new Array[T]|}

\subsection{Example}

\begin{figure}[H]
	\begin{minted}{scala}
		def checksum[T](data: Array[T]): Int = {
			val sum: Int = 0
			var index: Int = 0
			while (index < data.length) {
				val sum += data[i].##
				index += 1
			}
			
			return sum	
		}
	\end{minted}
	\caption{Example implementation of a checksum function.}
\end{figure}

\section{Specialization}

Cover the types of specializable terms.

\section{Specializing Methods}

\subsection{Typed Dispatch}

\begin{figure}[H]
	\begin{minted}{scala}
		def checksum(T: Type, data: Array[T]): Int = 
			if (T == Int)
				return checksum$Int(data.asInstanceOf[Array[Int]])
			 
			val sum: Int = 0
			var index: Int = 0
			while (index < data.length) {
				val sum += data[i].##
				index += 1
			}
			return sum	
		}
	\end{minted}
\end{figure}

\subsection{Code Duplication}

Specialized method for checksum
\begin{figure}[H]
	\begin{minted}{scala}
		def checksum$Int(data: Array[Int]): Int = {
			val sum: Int = 0
			var index: Int = 0
			while (index < data.length) {
				val sum += data[i] // hash code is identity for int
				index += 1
			}
			return sum	
		}
	\end{minted}
\end{figure}


\subsection{Partial Evaluation}


\begin{figure}
	\centering
	\includegraphics[width=0.5\columnwidth]{figures/checksum:Int:TruffleTier.png}
	\caption{Graal IR graph of specialization \texttt{checksum[Int]} after Truffle tier.}
\end{figure} 
%======================================================================
\chapter{Evaluation}

\begin{figure}[!htb]
	\begin{minted}{scala}
	class ArrayBuffer[T] {
		protected def initialSize: Int = 16
		var size0 = 0
		var array: Array[T] = newArray[T](Math.max(initialSize, 1))
		
		def length: Int = size0
		
		private def get(i: Int): T = array(i)
		private def set(i: Int, elem: T): Unit = array(i) = elem
		
		def contains(elem: T): Boolean = {
			var i = 0
			while (i < size0) {
				if (array(i) == elem)
				return true
				i += 1
			}
			false
		}
		
		def reverse(): Unit = {
			var pos = 0
			while (pos * 2 < size0) {
				val tmp1: T = array(pos)
				val tmp2: T = get(size0 - pos - 1)
				set(pos, tmp2)
				set(size0 - pos - 1, tmp1)
				pos += 1
			}
		}
		
		def append(elem: T): Unit = {
			val newSize0 = size0 + 1
			ensureSize(newSize0)
			set(size0, elem)
			size0 = newSize0
		}
		
		// Ensure that the internal array has at least `n` cells. 
		def ensureSize(n: Int): Unit = {
			// Use a Long to prevent overflows
			val arrayLength: Long = array.length 
			if (n > arrayLength) {
				var newSize: Long = arrayLength * 2
				while (n > newSize)
					newSize = newSize * 2
				// Clamp newSize to Int.MaxValue
				if (newSize > lang.Int.MaxValue) newSize = lang.Int.MaxValue
				
				val resized = newArray[T](newSize.toInt)
				var i = 0
				while (i < size0) {
					resized(i) = get(i)
					i += 1
				}
				array = resized
			}
		}
		// Swap two elements of this array, omitted for brevity
		def swap(a: Int, b: Int): Unit = ??? 
	\end{minted}
	\caption{Code of the \scalainline{ArrayBuffer} benchmark.}
	\label{example:arraybuffer-benchmark}
\end{figure}

%======================================================================

\chapter{Related Work}

\section{Truffle Interpreters}

\section{Specializing Scala}

\section{Specializing Other Languages}
%======================================================================
\chapter{Future Work}

\section{Profiling Type Arguments}

\section{Partial Specialization}

\section{Code Sharing}
%======================================================================
\chapter{Conclusions}

This thesis introduced \textsc{TastyTruffle}, a Truffle interpreter that is a platform for experimenting with ad-hoc data representations.
The thesis described methods to translate TASTy, a tree serialization format for Scala 3, into an executable IR that is suitable for execution in an optimizing interpreter.
We show in this thesis how to exploit the type information present in a input source language such as TASTy in order to generate specialized data representations for polymorphic data structures.
We demonstrate that these techniques can substantially improve the performance of simple Scala programs in an experimental when compared to a state-of-the-art Java virtual machine.

A particular challenge in the implementation of \textsc{TastyTruffle} was the translation of TASTy into \textsc{TastyTruffle} IR.
Because TASTy is emitted after parsing and type checking, no other compiler transformations typical in other intermediate representations are present.
Many features of the Scala programming language are built as abstractions of simpler constructs that must be further simplified by the compiler.
Without the existing compiler transformations to simplify these abstractions, TASTy can be at times \textit{extraneously} high-level for the purposes of execution.
While this did not significantly impact the evaluation of simple Scala programs for our experiments, it limits the \textit{breadth} of programs that are executable by our interpreter.
A possible solution to this hurdle is to read TASTy, perform a subset of Scala compiler transforms, then execute the program using our translation.
While we will have to avoid the type erasure transformation and all subsequent transformations which depend on the results of type erasure, a much larger portion of Scala programs become available for execution on our interpreter.

The specialization of classes with both class-polymorphic and method-polymorphic semantics proved to be a complex implementation detail.
The gap between the specialization of classes (at object creation) and the specialization of methods (at method invocation) required the selection of appropriate intermediate representation to encapsulate the \textit{partial} specialization.
Partial specializations have been specialized but also still contain polymorphic semantics which must resolved at a future specialization site.
In this thesis, we chose to use a high-level approach to aid the translation of TASTy definition with TASTy type arguments.
However, many approaches and mechanisms are possible to solve address this complexity.
TODO(accepting ideas)

In this thesis we have evaluated \textsc{TastyTruffle} on simple but nonetheless difficult to specialize data structures exhibiting bulk memory access as well random heap access.
The elimination of autoboxing in the list data structure resulted in incremental performance improvements where autoboxing proved to be a performance bottleneck.
The elimination of autoboxing in the context of data structures back by polymorphic arrays resulted in performance improvements by an order of magnitude.
\textsc{TastyTruffle} validates that there are many opportunities for data representation optimizations that bridge static compilation and just-in-time compilation.

%======================================================================

%----------------------------------------------------------------------
% END MATERIAL
%----------------------------------------------------------------------

% B I B L I O G R A P H Y
% -----------------------

% The following statement selects the style to use for references.  It controls the sort order of the entries in the bibliography and also the formatting for the in-text labels.
\bibliographystyle{plain}
% This specifies the location of the file containing the bibliographic information.  
% It assumes you're using BibTeX (if not, why not?).
\cleardoublepage % This is needed if the book class is used, to place the anchor in the correct page,
                 % because the bibliography will start on its own page.
                 % Use \clearpage instead if the document class uses the "oneside" argument
\phantomsection  % With hyperref package, enables hyperlinking from the table of contents to bibliography             
% The following statement causes the title "References" to be used for the bibliography section:
\renewcommand*{\bibname}{References}

% Add the References to the Table of Contents
\addcontentsline{toc}{chapter}{\textbf{References}}

\bibliography{uw-ethesis}
% Tip 5: You can create multiple .bib files to organize your references. 
% Just list them all in the \bibliogaphy command, separated by commas (no spaces).

% The following statement causes the specified references to be added to the bibliography% even if they were not 
% cited in the text. The asterisk is a wildcard that causes all entries in the bibliographic database to be included (optional).
\nocite{*}

% The \appendix statement indicates the beginning of the appendices.
\appendix
% Add a title page before the appendices and a line in the Table of Contents
\chapter*{APPENDICES}
\addcontentsline{toc}{chapter}{APPENDICES}

\chapter{Scala Unified Type System}
\includesvg[pretex=\footnotesize,width=\textwidth]{figures/svg/unified-types-diagram.svg}

\chapter{Scala 3 Compiler Phases}
\label{appendix:dotty-phases}
\begin{minted}{scala}
	/** Phases dealing with the frontend up to trees ready for TASTY pickling */
	protected def frontendPhases: List[List[Phase]] =
		List(new Parser) ::                       // scanner, parser
		List(new TyperPhase) ::                   // namer, typer
		List(new YCheckPositions) ::              // YCheck positions
		List(new sbt.ExtractDependencies) ::      // Sends information on classes' dependencies to sbt via callbacks
		List(new semanticdb.ExtractSemanticDB) :: // Extract info into .semanticdb files
		List(new PostTyper) ::                    // Additional checks and cleanups after type checking
		List(new sjs.PrepJSInterop) ::            // Additional checks and transformations for Scala.js (Scala.js only)
		List(new Staging) ::                      // Check PCP, heal quoted types and expand macros
		List(new sbt.ExtractAPI) ::               // Sends a representation of the API of classes to sbt via callbacks
		List(new SetRootTree) ::                  // Set the `rootTreeOrProvider` on class symbols
		Nil
\end{minted}

\begin{minted}{scala}
	/** Phases dealing with TASTY tree pickling and unpickling */
	protected def picklerPhases: List[List[Phase]] =
		List(new Pickler) ::            // Generate TASTY info
		List(new PickleQuotes) ::       // Turn quoted trees into explicit run-time data structures
		Nil
\end{minted}

\begin{minted}{scala}
	/** Phases dealing with the transformation from pickled trees to backend trees */
	protected def transformPhases: List[List[Phase]] =
		List(
			new FirstTransform,         // Some transformations to put trees into a canonical form
			new CheckReentrant,         // Internal use only: Check that compiled program has no data races involving global vars
			new ElimPackagePrefixes,    // Eliminate references to package prefixes in Select nodes
			new CookComments,           // Cook the comments: expand variables, doc, etc.
			new CheckStatic,            // Check restrictions that apply to @static members
			new BetaReduce,             // Reduce closure applications
			new init.Checker) ::        // Check initialization of objects
		List(
			new ElimRepeated,           // Rewrite vararg parameters and arguments
			new ExpandSAMs,             // Expand single abstract method closures to anonymous classes
			new ProtectedAccessors,     // Add accessors for protected members
			new ExtensionMethods,       // Expand methods of value classes with extension methods
			new UncacheGivenAliases,    // Avoid caching RHS of simple parameterless given aliases
			new ByNameClosures,         // Expand arguments to by-name parameters to closures
			new HoistSuperArgs,         // Hoist complex arguments of supercalls to enclosing scope
			new SpecializeApplyMethods, // Adds specialized methods to FunctionN
			new RefChecks) ::           // Various checks mostly related to abstract members and overriding
		List(
		 	// Turn opaque into normal aliases
			new ElimOpaque,            
			// Compile cases in try/catch
			new TryCatchPatterns,      
			// Compile pattern matches
			new PatternMatcher,         
			// Make all JS classes explicit (Scala.js only)
			new sjs.ExplicitJSClasses,  
			// Add accessors to outer classes from nested ones.
			new ExplicitOuter,          
			// Make references to non-trivial self types explicit as casts
			new ExplicitSelf,           
			// Expand by-name parameter references
			new ElimByName,             
			// Optimizes raw and s string interpolators by rewriting them to string concatentations
			new StringInterpolatorOpt) :: 
		List(
			new PruneErasedDefs,        // Drop erased definitions from scopes and simplify erased expressions
			new InlinePatterns,         // Remove placeholders of inlined patterns
			new VCInlineMethods,        // Inlines calls to value class methods
			new SeqLiterals,            // Express vararg arguments as arrays
			new InterceptedMethods,     // Special handling of `==`, `|=`, `getClass` methods
			new Getters,                // Replace non-private vals and vars with getter defs (fields are added later)
			new SpecializeFunctions,    // Specialized Function{0,1,2} by replacing super with specialized super
			new LiftTry,                // Put try expressions that might execute on non-empty stacks into their own methods
			new CollectNullableFields,  // Collect fields that can be nulled out after use in lazy initialization
			new ElimOuterSelect,        // Expand outer selections
			new ResolveSuper,           // Implement super accessors
			new FunctionXXLForwarders,  // Add forwarders for FunctionXXL apply method
			new ParamForwarding,        // Add forwarders for aliases of superclass parameters
			new TupleOptimizations,     // Optimize generic operations on tuples
			new LetOverApply,           // Lift blocks from receivers of applications
			new ArrayConstructors) ::   // Intercept creation of (non-generic) arrays and intrinsify.
		List(new Erasure) ::            // Rewrite types to JVM model, erasing all type parameters, abstract types and refinements.
		List(
			new ElimErasedValueType,    // Expand erased value types to their underlying implmementation types
			new PureStats,              // Remove pure stats from blocks
			new VCElideAllocations,     // Peep-hole optimization to eliminate unnecessary value class allocations
			new ArrayApply,             // Optimize `scala.Array.apply([....])` and `scala.Array.apply(..., [....])` into `[...]`
			new sjs.AddLocalJSFakeNews, // Adds fake new invocations to local JS classes in calls to `createLocalJSClass`
			new ElimPolyFunction,       // Rewrite PolyFunction subclasses to FunctionN subclasses
			new TailRec,                // Rewrite tail recursion to loops
			new CompleteJavaEnums,      // Fill in constructors for Java enums
			new Mixin,                  // Expand trait fields and trait initializers
			// Expand lazy vals
			new LazyVals,              
			// Add private fields to getters and setters
			new Memoize,                
			 // Expand non-local returns
			new NonLocalReturns,       
			// Represent vars captured by closures as heap objects
			new CapturedVars) ::        
		List(
			new Constructors,           // Collect initialization code in primary constructors
			// Note: constructors changes decls in transformTemplate, no InfoTransformers should be added after it
			new Instrumentation) ::     // Count calls and allocations under -Yinstrument
		List(
			// Lifts out nested functions to class scope, storing free variables in environments
			new LambdaLift,             
			// Note: in this mini-phase block scopes are incorrect. No phases that rely on scopes should be here
			// Replace `this` references to static objects by global identifiers
			new ElimStaticThis,         
			// Identify outer accessors that can be dropped
			new CountOuterAccesses) :: 
		List(
			// Drop unused outer accessors
			new DropOuterAccessors,     
			// Lift all inner classes to package scope
			new Flatten,                
			// Renames lifted classes to local numbering scheme
			new RenameLifted,           
			// Replace wildcards with default values
			new TransformWildcards,     
			 // Move static methods from companion to the class itself
			new MoveStatics,           
			// Widen private definitions accessed from nested classes
			new ExpandPrivate,          
			 // Repair scopes rendered invalid by moving definitions in prior phases of the group
			new RestoreScopes,         
			// get rid of selects that would be compiled into GetStatic
			new SelectStatic,           
			// Generate JUnit-specific bootstrapper classes for Scala.js (not enabled by default)
			new sjs.JUnitBootstrappers, 
			// Find classes that are called with super
			new CollectSuperCalls) ::   
		Nil
\end{minted}

\begin{minted}{scala}
	/** Generate the output of the compilation */
	protected def backendPhases: List[List[Phase]] =
		List(new backend.sjs.GenSJSIR) :: // Generate .sjsir files for Scala.js (not enabled by default)
		List(new GenBCode) ::             // Generate JVM bytecode
		Nil
\end{minted}
%======================================================================

\end{document}
