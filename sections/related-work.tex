
\chapter{Related Work}

This chapter discusses previous academic and industrial work related to this thesis. 
The first section provides an introduction on the various implementations of parametric polymorphism
The second section covers related work on the implementation on polymorphism in Java.
The third section of this chapter provides an overview on previous and state-of-the-art efforts to specialize Scala.
The last section presents prior and ongoing efforts in the implementation of other Truffle interpreters.

\section{Implementations of Parametric Polymorphism}

Implementations of parametric polymorphism can be divided into two broad categories\cite{java:odersky-type-params}:

\begin{description}
	\item[\textit{Homogeneous Translation}] 
	This approach provides a single data representation for each polymorphic type. 
	Examples of this implementation is the type erasure transformation applied in the Java and Scala compilation pipelines.
	Morrison et al. also refers to this form of polymorphism as the \scalainline{uniform polymorphism}\cite{types-of-polymorphism}.
	\item[\textit{Heterogeneous Translation}]
	In contrast to homogeneous translation, the heterogeneous translation ensures there is a unique data representation for every instantiation of a polymorphic type.
	The heterogeneous translation can also be referred to as \textit{textual polymorphism}.
\end{description}

In this section, we will cover various approaches in implementing parametric polymorphism in the context of these two forms.
As polymorphism in Java and Scala are more relevant to the central themes of this thesis, we will first focus on implementations of parametric polymorphism for other languages.

Parametric polymorphism was first studied in functional programming languages\cite{ml:parametric-polymorphism}\cite{ml:type-inference}.
Leroy proposed an approach where type coercion operations inserted between polymorphic operations and monomorphic data. 
The coercion operations in this approach is quite similar to the notion of boxing and unboxing, which Leroy describes as \textit{wrapping} and \textit{unwrapping}.

The heterogeneous translation is the more prevalent implementation of parametric polymorphism in object-oriented programming languages.
The \mintinline{cpp}|template| concept in the \CC\ programming language popularized parametric polymorphism in objected-oriented programming languages.
Templates define a generic definition of some kind in \CC.
The \CC\ compiler will generate heterogeneous translations based on every set of concrete type arguments supplied to during compilation.
The implementation of polymorphism in the Common Language Runtime\cite{clr:overview}\cite{clr:spec} by Kennedy and Syme makes use of reified types in a polymorphic bytecode IR during execution.
Polymorphic class definitions are loaded as templates; Templates generate specialize class layouts on an ad-hoc basis based on the reified type arguments seen during bytecode execution.
Their approach relies on extensions within the CLR to support types that is not present in existing JVM implementations.
Our approach shares many similarities with the approach described by Kennedy and Syme.
One drawback of their approach is the polymorphic bytecode IR does not support the full set of operations on types.
For example, reflection is necessary to differentiate between a \scalainline{List[Int]} and a \scalainline{List[String]}.
Our implementation differs as such operations are possible because the IR could potentially incorporate the full type language of TASTy.
 

\section{Generics and Java}

Prior efforts to implement generics in Java have been based on static compilation techniques restricted by the \textit{open world assumption}.
The open world assumption is an assumption that the program under compilation is \textit{incomplete}, extra parts of the program will be supplied in a future iteration of compilation.
This form of compilation is commonly known as \textit{separate compilation}.
As such, compilation results of the current parts of program must be interoperable with the compilation results of the remaining yet-to-be determined parts.

The Java language did not initially support parametric polymorphism in its initial release.
As a result, many different approaches were proposed before a uniform polymorphism became the accepted implementation for Java.
Pizza\cite{java:pizza} was a superset of Java that supported heterogeneous and homogeneous translations of polymorphic definitions into Java.
Agesen, Freund, and Mitchell proposed a heterogeneous translation for parametric polymorphism for Java during load-time instead of compile-time\cite{java:agesen-type-params}.
NextGen\cite{java:nextgen} separates the translation of polymorphic classes into monomorphic and polymorphic components.
In NextGen, Only the polymorphic members of a class definition are specialized; These specialized classes inherit the implementation of their monomorphic members from a common parent class.
Finally, GJ\cite{java:gj} proposed the foundations for what is now the accepted implementation of parametric polymorphism in Java.
Polymorphic class definitions have a single uniform data representation after type erasure.
All of these approaches determine the data representation of polymorphic definitions in a static context.
Our approach is based on the \textit{closed world assumption} as the entire program must be available in order for it to be executed.

\section{Specialization in Scala}

The standard implementation of parametric polymorphism follows that of Java, generic class definitions have their type parameters erased.
All previous approaches  attempt avoid the problem of bytecode explosion, where the specialization of polymorphic data with every possible type creates an exponential number of unique data representations.
Dragos describes the earliest efforts to specialize Scala programs with the aid of annotations\cite{scala:specialization}.
Annotations avoids unnecessarily specializing polymorphic data through knowledge injected by a programmer.
Ureche, Talau, and Odersky expand upon this approach by reducing unnecessary duplication among specializations through sharing\cite{scala:miniboxing}.
Sharing exploits the insight that specializations of some value types may be reused for the specializations of other value types.
For example, the representation of \scalainline{ArrayBuffer[Long]} could be used, with the addition of some glue code, for the specialization of \scalainline{ArrayBuffer[Int]} instead of generating an additional specialized representation. 
Both approaches mix the implementation of uniform polymorphism with user-guided specialization directives.
Our generates a heterogeneous translation of a generic class definition on an ad-hoc basis;

\section{Truffle Interpreters}

There many Truffle interpreters in active development at the time of writing.
In this section, we will attempt to provide a brief survey of Truffle interpreters.
TruffleRuby\cite{trufflyruby:specialization}\cite{truffleruby:object-model},FastR, Graal.js, Graal.Python,\cite{truffle:thesis} are some of the industrial implementations of dynamically typed languages implemented with Truffle.
They all make substantial use of Truffle facilities, some discussed earlier in this thesis, to speculative optimize program execution. 
Espresso\cite{graalvm:espresso} is an implementation of a Java bytecode interpreter in Truffle. 
Espresso is a metacircular implementation of a Java Virtual Machine.
Because Espresso executes the same Java bytecode format as other JVM implementations, it makes use of same approaches to optimizing polymorphic data layout as the conventional implementation of Java on GraalVM.


