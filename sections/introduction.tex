\chapter{Introduction}

\acrfull{jit} compilation has seen great success in implementing runtimes for objected-oriented programming languages.
It has effectively generated efficient machine code in the presence of virtual dispatch arising from \textit{subtype} polymorphism.
While a call site may statically have many possible call targets, JIT compilation can incorporate dynamic runtime information to optimize the most frequently invoked call targets speculatively.
These speculative optimizations often enable compiled code to be inlined, a critical transformation in the context of JIT compilation.
Inlining compiled code generates opportunities for many further optimizations.

Many object-oriented languages have since incorporated the notion of generic programming, otherwise known as \textit{parametric} polymorphism.
Parametric polymorphism enables programs to be more modular and reusable as functions and data structures behave identically\cite{tapl} regardless of the types of their inputs.
Implementations of generic programming often come at the expense of program complexity and performance.
Static compilers for object-oriented languages with parametric polymorphism must compromise when selecting an appropriate data representation for polymorphic data types and functions.
This trade-off comes down to more optimal data layouts at the expense of space or uniform data layouts, which are not optimal for every type at the expense of performance.

The selection of an optimal data representation, or \textit{specialization}, of a polymorphic data structure relies on information typically found in the type-rich source language of programming languages.
Representations must be consistent throughout the whole program as code that manipulates such data structures assume their representations to be consistent.
Consequently, the specialization problem is best suited to compilers with access to whole program information during compilation.
However, this is not the case for object-oriented languages such as Java and Scala, which statically generate a uniform data representation for their polymorphic definitions to guarantee consistency throughout the whole program. 
Additionally, static compilers do not have sufficient runtime information, which is critical in making favourable optimization decisions compared to JIT compilers.
On the other hand, JIT compilers are ill-suited to whole program optimizations as they are best at the dynamic optimization of small regions of a program.
Therefore, the problem of specialization falls between static compilation and JIT compilation.

This thesis introduces \textsc{TastyTruffle}, an interpreter and JIT compiler which incorporates rich source-level type information with speculative optimizations to specialize data representations for the Scala programming language.
\textsc{TastyTruffle} is implemented in Truffle, a framework that simplifies the implementation of a JIT compiler for a source language by implementing an interpreter for that language. 
Our source language is the \acrfull{tasty} serialization format emitted by the Scala 3 compiler.
TASTy is an abstract syntax tree format emitted after parsing and type checking Scala programs.
By using TASTy, a suitable source language, we can access source-level type information without having to parse and type check a Scala source program.

The contributions of this thesis are as follows: 
\begin{enumerate}
	\item The implementation of an interpreter for the TASTy format using Truffle and the necessary transformations to make a TASTy program executable. TASTy is high-level uncanonical representation of Scala not suitable for execution; Non-trivial transformations must be applied to a TASTy program before execution.
	In contrast, Java bytecode of compiled Scala programs is readily available for execution on any Java virtual machine.
	\item The extension of interpreter to support specialized data representations of generic types.
	These specialized data representations are created using concrete type arguments that generic types are instantiated with.
	with generic data structures.
	\item The evaluation of the interpreter on simple and realistic programs that present a challenge to existing state-of-the-art techniques.
\end{enumerate}

\newpage

\section{Thesis Organization}

We describe the layout of the remainder of this thesis.
Chapter 2 provides an overview of the many intermediate representations of Scala from compilation to execution.
It explores the advantages and drawbacks of each intermediate representation concerning specialization.
Chapter 3 details the implementation of \textsc{TastyTruffle}.
It covers the translation of TASTy into a more suitable IR for execution in an interpreter where each polymorphic data structure has a uniform representation.
The chapter then provides extensions to the interpreter to support the just-in-time specialization of polymorphic data structures.
Chapter 4 evaluates the interpreter with and without extensions for dynamic specialization on simple but realistic data structures.
The chapter provides the performance of these evaluated data structures in the context of the standard implementation of Scala with the underlying JIT compiler of our interpreter without any augmentation.
Chapter 5 explores related work in various implementations of parametric polymorphism and other Truffle interpreters.
Chapter 6 discusses possible extensions to \textsc{TastyTruffle} to better integrate source-level type semantics with JIT compilation.
Chapter 7 concludes the thesis.