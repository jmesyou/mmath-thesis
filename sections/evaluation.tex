\chapter{Evaluation}

In this chapter, we will evaluate and discuss the performance of our polymorphic interpreter on two benchmarks.
We use an existing set of benchmarks from \cite{scala:miniboxing} as they exercise many features of the Scala runtime that require specialization to perform optimally.
We will evaluate performance of these benchmarks on the monomorphic interpreter as well as Scala bytecode on GraalVM as points of comparison for relative performance.
Finally, we will discuss the results of the benchmarks.

\section{Benchmarks}
\begin{figure}[!htb]
	\begin{minted}{scala}
	class ArrayBuffer[T] {
		protected def initialSize: Int = 16
		var size0 = 0
		var array: Array[T] = newArray[T](Math.max(initialSize, 1))
		
		def length: Int = size0
		
		private def get(i: Int): T = array(i)
		private def set(i: Int, elem: T): Unit = array(i) = elem
		
		def contains(elem: T): Boolean = {
			var i = 0
			while (i < size0) {
				if (array(i) == elem) return true
				i += 1
			}
			false
		}
		
		def reverse(): Unit = {
			var pos = 0
			while (pos * 2 < size0) {
				swap(pos, size0 - pos - 1) // swaps two elements in the array
				pos += 1
			}
		}
		
		def append(elem: T): Unit = {
			val newSize0 = size0 + 1
			ensureSize(newSize0)
			set(size0, elem)
			size0 = newSize0
		}
		
		// Ensure that the internal array has at least `n` cells. 
		def ensureSize(n: Int): Unit = {
			val arrayLength: Long = array.length // Use a Long to prevent overflows
			if (n > arrayLength) {
				var newSize: Long = arrayLength * 2
				while (n > newSize)
					newSize = newSize * 2
				// Clamp newSize to Int.MaxValue
				if (newSize > lang.Int.MaxValue) newSize = lang.Int.MaxValue
				
				val resized = newArray[T](newSize.toInt)
				var i = 0
				while (i < size0) {
					resized(i) = get(i)
					i += 1
				}
				array = resized
			}
		}
	\end{minted}
	\caption{Code of the \scalainline{ArrayBuffer} benchmark.}
	\label{example:arraybuffer-benchmark}
\end{figure}

In this section, we will introduce an additional program on top of our running example for benchmarking.
We will also summarize the motivations from \cite{scala:miniboxing} for the selection of these benchmarks.

\section{Methodology}

Performance measurement of just-in-time compiled programs are is infamously difficult issue\cite{java:performance-analysis}\cite{java:statistically-rigor-performance-analysis}.
Many non-deterministic effects, such as speculative optimization, garbage collection, thread scheduling to name a few, affect the performance of programs executing on the Java Virtual Machine.
As result, the JVM must be \textit{warmed up} prior to measure of program performance.
A benchmarking routine is warmed up with several iterations of invocations in order for profiling data to be collected and JIT compilation to finished.
Therefore the measured performance of a microbenchmark will record the program executing the stable JIT compiled code instead of code executing in the interpreter.

Each benchmark method in this chapter is warmed up with $10$ iterations of warmup lasting $10$ seconds each.
Results of these is measured in throughput, the number of executions that successfully completed in a second.
The results are averaged from $10$ measurements iterations for a period of $10$ seconds each.
We evaluate our microbenchmarks on input sizes between one hundred thousand and one million elements to account for factors of memory in our benchmarks. 
Each benchmark is run on three different implementations, Scala on GraalVM (Graal), the monomorphic interpreter (Mono), and the polymorphic interpreter (Poly).

\section{Experimental Results}

\begin{figure}[!htb]
	\centering
	\includesvg[width=\textwidth]{data/ArrayBuffer.Append.svg}
	\caption{Benchmark results for \scalainline{ArrayBuffer.append}.}
\end{figure}

The benchmark for \scalainline{ArrayBuffer.append} inserts a sequence of elements into a newly initialized array buffer. 
This benchmark stresses array memory movement.
Each time the backing array is too small for an additional element, the backing array is resized by creating a new larger and copying over existing elements.
This resizing operation (\scalainline{ensureSize} in \ref{example:arraybuffer-benchmark}) dominates the time spent in execution.
Because of this bottleneck, executing compiled Scala bytecode on GraalVM is up to 4 times faster than the monomorphic and polymorphic interpreter.

\begin{figure}[!htb]
	\centering
	\includesvg[width=\textwidth]{data/ArrayBuffer.Contains.svg}
	\caption{Benchmark results for \scalainline{ArrayBuffer.contains}.}
\end{figure}

The \scalainline{ArrayBuffer.contains} benchmark tests array operations in isolation. 
The benchmark checks an array buffer for the existence of a element.
It exercises a polymorphic array access as well as 

\begin{figure}[!htb]
	\centering
	\includesvg[width=\textwidth]{data/ArrayBuffer.Reverse.svg}
	\caption{Benchmark results for \scalainline{ArrayBuffer.reverse}.}
\end{figure}

\scalainline{ArrayBuffer.reverse} reverses the order of the elements in the array buffer.
Reversing an array is performance-bound by the loop of swap operations.
A swap operation (given in \ref{impl:swap}) consists two polymorphic value definitions initialized from polymorphic array accesses followed by the inverse of those two operations.

\begin{figure}[!htb]
\begin{minted}{scala}
	def swap(i: Int, j: Int): Unit = {
		val tmp1: T = get(i)
		val tmp2: T = get(j)
		set(i, tmp2)
		set(j, tmp1)
	}
\end{minted}
\caption{Code to swap two elements in an array buffer}
\label{impl:swap}
\end{figure}

The optimization of this microbenchmark proved to be the most difficult benchmark in terms of matching hand written monomorphic code in \cite{scala:miniboxing}.
The performance between the monomorphic interpreter and GraalVM is roughly equal between all types; 
Neither implementation is able to specialize the polymorphic reads and array accesses.
The polymorphic interpreter has up to $25$ times more throughput than the monomorphic interpreter and GraalVM.

\begin{figure}[!htb]
	\centering
	\includesvg[width=\textwidth]{data/List.Append.svg}
	\caption{Benchmark results for \scalainline{List.append}.}
\end{figure}

The \scalainline{List.append} benchmark constructs a list from an array.
As the creation of polymorphic instances is predominantly memory-bound and not compute-bound, there is no significant improvement throughput from specialization.
In fact, executing Scala via Java bytecode on the JMV results in substantially greater throughput.

\begin{figure}[!htb]
	\centering
	\includesvg[width=\textwidth]{data/List.Contains.svg}
	\caption{Benchmark results for \scalainline{List.contains}.}
\end{figure}

Like \scalainline{ArrayBuffer.contains}

\begin{figure}[!htb]
	\centering
	\includesvg[width=\textwidth]{data/List.Hashcode.svg}
	\caption{Benchmark results for \scalainline{List.hashCode}.}
\end{figure}

\scalainline{List.hashCode}