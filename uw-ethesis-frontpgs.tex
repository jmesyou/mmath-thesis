% T I T L E   P A G E
% -------------------
% Last updated June 14, 2017, by Stephen Carr, IST-Client Services
% The title page is counted as page `i' but we need to suppress the
% page number. Also, we don't want any headers or footers.
\pagestyle{empty}
\pagenumbering{roman}

% Source-guided Ad-hoc Data Representations in a Managed Runtime
% The contents of the title page are specified in the "titlepage"
% environment.
\begin{titlepage}
        \begin{center}
        \vspace*{1.0cm}

        \Huge
        {\bf Specializing Scala with Truffle}

        \vspace*{1.0cm}

        \normalsize
        by \\

        \vspace*{1.0cm}

        \Large
        James You \\

        \vspace*{3.0cm}

        \normalsize
        A thesis \\
        presented to the University of Waterloo \\ 
        in fulfillment of the \\
        thesis requirement for the degree of \\
        Master of Mathematics \\
        in \\
        Computer Science \\

        \vspace*{2.0cm}

        Waterloo, Ontario, Canada, 2022 \\

        \vspace*{1.0cm}

        \copyright\ James You 2022 \\
        \end{center}
\end{titlepage}

% The rest of the front pages should contain no headers and be numbered using Roman numerals starting with `ii'
\pagestyle{plain}
\setcounter{page}{2}

\cleardoublepage % Ends the current page and causes all figures and tables that have so far appeared in the input to be printed.
% In a two-sided printing style, it also makes the next page a right-hand (odd-numbered) page, producing a blank page if necessary.

% D E C L A R A T I O N   P A G E
% -------------------------------
  % The following is a sample Delaration Page as provided by the GSO
  % December 13th, 2006.  It is designed for an electronic thesis.
  \noindent
I hereby declare that I am the sole author of this thesis. This is a true copy of the thesis, including any required final revisions, as accepted by my examiners.

  \bigskip
  
  \noindent
I understand that my thesis may be made electronically available to the public.

\cleardoublepage

% A B S T R A C T
% ---------------

\begin{center}\textbf{Abstract}\end{center}

Scala is a generic object-oriented programming language with higher-order abstractions. 
Programming abstractions in Scala exemplify reusability and extensibility in the context of type safety.
In particular, generic programming allows user-defined data structures to behave identically irrespective of the types of their values while remaining free of type errors.

The implementation of reusability in Scala comes at a cost; The standard implementation of Scala compiles to Java bytecode, where type erasure significantly reduces Scala program type information to create compatible Java bytecode.
Consequently, autoboxing, operations needed when using primitive values in a generic context, are unnecessarily introduced into the final program. 
The current state-of-the-art techniques for eliminating boxing and achieving optimal data representations at runtime, known as specialization, rely on static program analysis.
Such techniques must mitigate the problem of code duplication as static optimizations cannot use runtime information to best select which data structures to specialize.

We propose a new approach to the specialization of Scala programs.
Our approach integrates type information from a high-level source-like input language with the mechanisms of just-in-time compilation.
We propose an ad-hoc specialization mechanism using a whole program approach; Specializations of data structures are created based on concrete type arguments.
In our approach, specialized objects are compatible with non-specialized code.
We use Truffle, a framework that simplifies the implementation of interpreters and just-in-time compilers, to implement an experimental research prototype.

We demonstrate that our approach is viable and produces improvements in throughput for simplified implementations of real-world Scala programs.
While these programs are simple, it is still challenging for state-of-the-art approaches to specialize optimally.
We show that our approach can improve performance by an order of magnitude in the context of polymorphic data structures and methods that use bulk storage.
We compare the results of our approach to our interpreter without specialization and compiled Scala on GraalVM, a state-of-art Java Virtual Machine.

\cleardoublepage

% A C K N O W L E D G E M E N T S
% -------------------------------

\begin{center}\textbf{Acknowledgements}\end{center}

I would like to thank all the little people who made this thesis possible.
\cleardoublepage

% D E D I C A T I O N
% -------------------

\begin{center}\textbf{Dedication}\end{center}

This is dedicated to the one I love.
\cleardoublepage

% T A B L E   O F   C O N T E N T S
% ---------------------------------
\renewcommand\contentsname{Table of Contents}
\tableofcontents
\cleardoublepage
\phantomsection    % allows hyperref to link to the correct page

% L I S T   O F   T A B L E S
% ---------------------------
\addcontentsline{toc}{chapter}{List of Tables}
\listoftables
\cleardoublepage
\phantomsection		% allows hyperref to link to the correct page

% L I S T   O F   F I G U R E S
% -----------------------------
\addcontentsline{toc}{chapter}{List of Figures}
\listoffigures
\cleardoublepage
\phantomsection		% allows hyperref to link to the correct page

% GLOSSARIES (Lists of definitions, abbreviations, symbols, etc. provided by the glossaries-extra package)
% -----------------------------
\printglossaries
\cleardoublepage
\phantomsection		% allows hyperref to link to the correct page

% Change page numbering back to Arabic numerals
\pagenumbering{arabic}

