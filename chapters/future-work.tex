\chapter{Future Work}

TastyTruffle is intended to be a framework for dynamic whole-program approaches to optimizing Scala.
In this section, we discuss some possible extensions to the interpreter that further take advantage of Truffle mechanisms.
A substantial penalty of heterogeneous translations of polymorphic programs is \textit{code explosion}.
For large polymorphic programs, the penalty of heterogeneous translation is twofold.
First, the cost of increased memory usage.
Having many specialized data representations incurs extra storage unless these data representations are regenerated every time a specialization is needed.
The second is the hidden computational overhead of specialization. 
Like other computational overheads of managed runtimes such as garbage collection, time spent generating specialized variants of polymorphic classes or methods means time not spent executing the program.
This overhead may be reduced by distributing such workloads into multiple threads.
We propose several heuristics to influence \textit{when} a specialization is created.

Many prior approaches to specialization have already attempted to minimize the number of specializations to mitigate performance degradation for complex polymorphic definitions, where there is often a \(O(t^n)\) space complexity worse case\footnote{$t$ is the number of values types combined with the reference type, $n$ is the number of type parameters in a generic definition.}.
These approaches balance the tradeoff between performance and code size to optimistically generate \textit{only} the specializations required to eliminate performance bottlenecks.
Because of the work done in \cite{scala:specialization}, many existing Scala programs are already user-annotated with a specialization directive.
Similarly, our approach could be extended to include the semantics of this annotation, generating specializations with user-guide information only where needed.
The translation of non-annotated definitions will use a shared type-erased data representation.
However, mixing annotated and non-annotated programs will present missed optimization opportunities in the non-annotated portions of the program.

Truffle offers many mechanisms for profiling values and types.
Some of this profiling instrumentation is automatically done by Truffle, such as the profiling of argument types for node rewrites. In contrast, the guest language implementer adds other instrumentation,  such as condition profiles.
These profiles augment partial evaluation and enable speculative assumptions for optimizations such as conditional elimination.
We propose instrumenting specialization sites to profile type arguments.
Type argument profiles could then be used to decide the specific instantiation to specialize.
A \textit{profile-guided} approach to specialization could limit specializations to only the most frequently used instantiations.

Often a polymorphic instantiation is not sufficiently frequent to warrant specialization; the default homogeneous data representation will be shared among unspecialized instantiations.
A type-erased homogeneous data representation may still be tagged with the underlying applied type.
We can further augment type-erased polymorphic fields, and frame slots to profile reads from and writes to their respective storage locations.
These two pieces of dynamic information can be combined to allow the specialization of specific instantiations that are frequently manipulated but not frequently created.

We can apply the same optimization to sharing data layouts between specific specializations with inspiration from the work done by Ureche et al. in \cite{scala:miniboxing}.
Because the additional operations that adapt shared specializations to their original type contexts are negligible in terms of performance\cite{scala:miniboxing}, these operations make sense to implement directly as Truffle nodes that will be further optimized by JIT compilation.